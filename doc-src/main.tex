\documentclass[12pt]{article}
\usepackage[czech]{babel}
\usepackage{natbib}
\usepackage{url}
\usepackage[utf8x]{inputenc}
\usepackage{amsmath}
\usepackage{graphicx}
\graphicspath{{images/}}
\usepackage{parskip}
\usepackage{fancyhdr}
\usepackage{vmargin}
\setmarginsrb{3 cm}{2.5 cm}{3 cm}{2.5 cm}{1 cm}{1.5 cm}{1 cm}{1.5 cm}
\usepackage{fancyhdr}
\usepackage{caption}
\usepackage{array}
\newcolumntype{?}[1]{!{\vrule width #1}}
\usepackage{hyperref}
\hypersetup{
    colorlinks=true,
    linkcolor=red,
    filecolor=red,
    urlcolor=blue,
    citecolor=red,
    linktoc=none
    }


\makeatletter
\let\thetitle\@title


\makeatother

\pagestyle{fancy}
\fancyhf{}
\rhead{\theauthor}
\lhead{\thetitle}
\cfoot{\thepage}


%%%%%%%%%%%%%%%%%%%%%%%%%%%%%%%%%%%%%%%%%%%%%%%%%%%%%%%%%%%%%%%%%%%%%%%%%%%%%%%%%%%%%%%%%
\begin{titlepage}
	\centering
    \includegraphics[scale = 0.35]{FIT_logo.png}\\[1.0 cm]	% University Logo
    \textsc{\LARGE Projektová dokumentace} \\[2.0 cm]
    \textsc{\LARGE Implementace systému pro získání, zpracování a ukládání dat} \\[1 cm]
    \textbf{\LARGE COVID-19}
	\rule{\linewidth}{0.2 mm} \\[0.4 cm]
	{ \huge \bfseries \thetitle}\\

	\vspace{1 cm}
	\begin{minipage}{0.45\textwidth}

            \newline
			\begin{flushleft}
			\emph{Autoři:} \\
			\textbf{Tomáš Beránek (xberan46)} \linebreak
			\textbf{Daniel Kamenický (xkamen21)} \linebreak
			\textbf{Richard Klem (xklemr00)} \linebreak

			% Your Student Number
		    \end{flushleft}
	\end{minipage}\\[0 cm]

    \vspace{8 cm}
    \begin{flushleft}
        Datum odevzdání:\hspace{2 cm}\textbf{3. listopadu 2021} \linebreak
    \end{flushleft}

	\vfill
    \fancyhf{}
    \fancyhead[R]{3. listopadu 2021}
\end{titlepage}

%%%%%%%%%%%%%%%%%%%%%%%%%%%%%%%%%%%%%%%%%%%%%%%%%%%%%%%%%%%%%%%%%%%%%%%%%%%%%%%%%%%%%%%%%

\renewcommand{\contentsname}{Obsah}
\tableofcontents
\pagebreak

%%%%%%%%%%%%%%%%%%%%%%%%%%%%%%%%%%%%%%%%%%%%%%%%%%%%%%%%%%%%%%%%%%%%%%%%%%%%%%%%%%%%%%%%%
\begin{document}
\afterpage{\cfoot{\thepage }}

\section{Řešení 1. části projektu}
Cílem 1. části projektu je:
\begin{itemize}
    \item analýza zdrojů dat a jejich dílčích datových sad, zejména struktura dat, atributy, datové typy atributů, možnosti propojení datových sad atp.,
    \item vytvoření systému pro stažení, předzpracování a vložení dat do MongoDB.
\end{itemize}

\subsection{Volba dotazů ze zadání}
\label{dotazy}
Pro vhodné předzpracování dat je dobré znát způsob, jakým se budou data využívat. Na základě této informace je pak možné například -- identifikovat nepotřebná data, která lze v předzpracování odstranit, zvolit vhodný databázový systém, upravit formát atributů v předzpracování atp. Cílem 2. části projektu je realizace dvou dotazů typu A, jednoho dotazu typu B, dvou vlastních dotazů a příprava dat pro jednu z dolovacích úloh typu C.

\textbf{Vybrané dotazy typu A}:
\begin{itemize}
    \item[1)] Vytvořte čárový (spojnicový) graf zobrazující vývoj covidové situace po měsících pomocí následujících hodnot: počet nově nakažených za měsíc, počet nově vyléčených za měsíc, počet nově hospitalizovaných osob za měsíc, počet provedených testů za měsíc. Pokud nebude výsledný graf dobře čitelný, zvažte logaritmické měřítko, nebo rozdělte hodnoty do více grafů.

    \item[] \textbf{Potřebná data}:
    \begin{itemize}
        \item \textbf{počet nově nakažených} (minimální granularita -- měsíc)
        \item \textbf{počet nově vyléčených} (minimální granularita -- měsíc)
        \item \textbf{počet nově hospitalizovaných} (minimální granularita -- měsíc)
        \item \textbf{počet nově provedených testů} (minimální granularita -- měsíc)
    \end{itemize}

    \hspace{1cm}

    \item[3)] Vytvořte sérii sloupcových grafů, které zobrazí:
    \begin{itemize}
        \item počty provedených očkování v jednotlivých krajích (celkový počet od začátku očkování),
        \item počty provedených očkování jako v předchozím bodě navíc rozdělené podle pohlaví. Diagram může mít např. dvě části pro jednotlivá pohlaví,
        \item počty provedených očkování, ještě dále rozdělené dle věkové skupiny. Pro potřeby tohoto diagramu postačí 3 věkové skupiny (0-24 let, 25-59 let, nad~59).
    \end{itemize}

    \item[] \textbf{Potřebná data}:
    \begin{itemize}
        \item \textbf{počet provedených očkovaní u můžu/žen v daném kraji} (minimální granularita -- celkem)
        \item \textbf{počet provedených očkovaní podle věku v daném kraji} (minimální granularita počtu očkování -- celkem, minimální granularita věku -- 0-24 let, 25-59 let a nad 59 let)
    \end{itemize}
\end{itemize}

\hspace{1cm}

\textbf{Vybraný dotaz typu B}:
\begin{itemize}
    \item[2)] Vytvořte sérii sloupcových grafů (alespoň 3), které porovnají vývoj různých covidových ukazatelů vámi zvoleného kraje se zbytkem republiky. Jako covidové ukazatele můžete použít: počet nakažených osob, počet hospitalizovaných osob, počet zemřelých, počet očkovaných. Všechny hodnoty uvažujte přepočtené na jednoho obyvatele kraje/republiky. Zobrazte alespoň 12 po sobě jdoucích hodnot (např. hodnoty za poslední rok po měsících).
\end{itemize}

    \item[] \textbf{Potřebná data}:
    \begin{itemize}
        \item \textbf{celkový počet nakažených podle kraje} (minimální granularita -- měsíc)
        \item \textbf{celkový počet zemřelých podle kraje} (minimální granularita -- měsíc)
        \item \textbf{celkový počet očkovaných podle kraje} (minimální granularita -- měsíc)
    \end{itemize}

\hspace{1cm}

\textbf{Vlastní dotazy} zahrnující data alespoň ze dvou zdrojů:
\begin{itemize}
    \item[1)] Vytvořte sloupcový graf rozdělený podle věku (po 5letých intervalech) obyvatelstva ve vybraném okresu. Každý sloupec vyjadřuje kolik procent obyvatelstva z dané věkové skupiny dostalo COVID-19 za poslední rok (neuvažujte duplicity).

    \item[] \textbf{Potřebná data}:
    \begin{itemize}
        \item \textbf{počet nakažených podle věku a okresu} (minimální granularita věku -- 5leté intervaly)
        \item \textbf{počet obyvatel podle věku a okresu} (minimální granularita věku -- 5leté intervaly)
    \end{itemize}

    \item[2)] Vytvořte graf, který ukazuje vývoj počtu nakažených žen a mužů začátku pandemie COVID-19 po měsících. Hodnoty uvádějte v procentech (např. 100% u mužů znamená, že jsou nakaženi všichni muži).

    \item[] \textbf{Potřebná data}:
    \begin{itemize}
        \item \textbf{počet nakažených podle pohlaví } (minimální granularita -- měsíc)
    \end{itemize}
\end{itemize}

\hspace{1cm}

\textbf{Vybraná dolovací úloha typu C}:
\begin{itemize}
    \item[1)] Hledání skupin podobných okresů z hlediska vývoje covidu a věkového složení obyvatel.
    \begin{itemize}
        \item Atributy: počet nakažených za poslední 4 čtvrtletí, počet očkovaných za poslední 4 čtvrtletí, počet obyvatel ve věkové skupině 0..14 let, počet obyvatel ve věkové skupině 15 - 59, počet obyvatel nad 59 let.
        \item Pro potřeby projektu vyberte libovolně 50 okresů, pro které najdete potřebné hodnoty (můžete např. využít nějaký žebříček 50 nejlidnatějších okresů v ČR).
    \end{itemize}

    \item[] \textbf{Potřebná data}:
    \begin{itemize}
        \item \textbf{počet nakažených podle věku a okresů} (minimální granularita věku -- 0-24 let, 25-59 let a nad 59 let, minimální granularita datumu -- 3 měsíce)
        \item \textbf{počet očkovaných podle věku a okresů} (minimální granularita věku -- 0-24 let, 25-59 let a nad 59 let, minimální granularita datumu -- 3 měsíce)
        \item \textbf{počty obyvatel v jednotlivých okresech podle věku} (minimální granularita věku -- 0-24 let, 25-59 let a nad 59 let)
    \end{itemize}
\end{itemize}

\newpage
\subsection{Analýza datových sad}
Součástí 1. části projektu je i analýza datových sad. Zdroje dat, které je možno využít jsou definovány zadáním. Každý zdroj poskytuje řadu dílčích datových sad. Následující seznam obsahuje pouze datové sady, které poskytují potřebná data uvedená v kapitole \ref{dotazy}. Pro každou dílčí datovou sadu jsou uvedeny informace:
\begin{itemize}
    \item \textbf{URL} -- ze které je možno datovou sadu stáhnout,
    \item \textbf{formát} -- v jakém formátu jsou data uložena např. CSV, JSON, HTML atp.,
    \item \textbf{struktura} -- uvedeno tabulkou.
\end{itemize}

\subsubsection{COVID-19 v ČR: Otevřené datové sady}
\label{data1}
Zdroj je dostupný \href{https://onemocneni-aktualne.mzcr.cz/api/v2/covid-19}{zde}. Celkem poskytuje 15 dílčích datových sad. V tomto řešení projektu preferujeme formát CSV, proto jsou jako odkazy na dílčí datové sady uvedeny odkazy přímo na data ve formátu CSV. Důvod preferování výběru CSV formátu je uveden v kapitole \ref{predzpracovani}.


\subsubsection*{Přehled osob s prokázanou nákazou dle hlášení krajských hygienických stanic (v2)}
Potřebné pro dotaz A1, B2. C1, vlastní dotaz 1 a vlastní dotaz 2.
\begin{itemize}
    \item \textbf{formát}: CSV, JSON.
    \item \textbf{dostupné  \href{https://onemocneni-aktualne.mzcr.cz/api/v2/covid-19/osoby.csv}{zde}}
    \item \textbf{struktura}:
\end{itemize}
        \begin{center}
            \begin{tabular}{ |l|c|c| }
                \hline
                \multicolumn{1}{|c|}{Jméno} & Datový typ & Užitečné \\
                \hline
                \hline
                datum & date & ano \\
                \hline
                vek & integer & ano \\
                \hline
                pohlavi & string & ano \\
                \hline
                kraj\_nuts\_kod & string & ano \\
                \hline
                okres\_lau\_kod & string & ano \\
                \hline
                nakaza\_v\_zahranici & boolean & ne \\
                \hline
                nakaza\_zeme\_csu\_kod & string & ne \\
                \hline
            \end{tabular}
        \end{center}

\newpage
\subsubsection*{Přehled vyléčených dle hlášení krajských hygienických stanic}
Potřebné pro dotaz A1.
\begin{itemize}
    \item \textbf{formát}: CSV, JSON.
    \item \textbf{dostupné  \href{https://onemocneni-aktualne.mzcr.cz/api/v2/covid-19/vyleceni.csv}{zde}}
    \item \textbf{struktura}:
\end{itemize}
        \begin{center}
            \begin{tabular}{ |l|c|c| }
                \hline
                \multicolumn{1}{|c|}{Jméno} & Datový typ & Užitečné \\
                \hline
                \hline
                datum & date & ano \\
                \hline
                vek & integer & ne \\
                \hline
                pohlavi & string & ne \\
                \hline
                kraj\_nuts\_kod & string & ne \\
                \hline
                okres\_lau\_kod & string & ne \\
                \hline
            \end{tabular}
        \end{center}

\subsubsection*{Přehled hospitalizací}
Potřebné pro dotaz A1.
\begin{itemize}
    \item \textbf{formát}: CSV, JSON.
    \item \textbf{dostupné \href{https://onemocneni-aktualne.mzcr.cz/api/v2/covid-19/hospitalizace.csv}{zde}}
    \item \textbf{struktura}:
\end{itemize}
        \begin{center}
            \begin{tabular}{ |l|c|c| }
                \hline
                \multicolumn{1}{|c|}{Jméno} & Datový typ & Užitečné \\
                \hline
                \hline
                datum & date & ano \\
                \hline
                pacient\_prvni\_zaznam & integer & ne \\
                \hline
                kum\_pacient\_prvni\_zaznam & integer & ne \\
                \hline
                pocet\_hosp & integer & ano \\
                \hline
                stav\_bez\_priznaku & integer & ne \\
                \hline
                stav\_lehky & integer & ne \\
                \hline
                stav\_stredni & integer & ne \\
                \hline
                stav\_tezky & integer & ne \\
                \hline
                jip & integer & ne \\
                \hline
                kyslik & integer & ne \\
                \hline
                hfno & integer & ne \\
                \hline
                upv & integer & ne \\
                \hline
                ecmo & integer & ne \\
                \hline
                tezky\_upv\_ecmo & integer & ne \\
                \hline
                umrti & integer & ne \\
                \hline
                kum\_umrti & integer & ne \\
                \hline
            \end{tabular}
        \end{center}

\subsubsection*{Celkový (kumulativní) počet provedených testů podle krajů a okresů ČR}
Potřebné pro dotaz A1.
\begin{itemize}
    \item \textbf{formát}: CSV, JSON.
    \item \textbf{dostupné  \href{https://onemocneni-aktualne.mzcr.cz/api/v2/covid-19/kraj-okres-testy.csv}{zde}}
    \item \textbf{struktura}:
\end{itemize}
        \begin{center}
            \begin{tabular}{ |l|c|c| }
                \hline
                \multicolumn{1}{|c|}{Jméno} & Datový typ & Užitečné \\
                \hline
                \hline
                datum & date & ano \\
                \hline
                kraj\_nuts\_kod & string & ano \\
                \hline
                okres\_lau\_kod & string & ne \\
                \hline
                prirustkovy\_pocet\_testu\_okres & integer & ne \\
                \hline
                kumulativni\_pocet\_testu\_okres & integer & ne \\
                \hline
                prirustkovy\_pocet\_testu\_kraj & integer & ano \\
                \hline
                kumulativni\_pocet\_testu\_kraj & integer & ne \\
                \hline
                prirustkovy\_pocet\_prvnich\_testu\_okres & integer & ne \\
                \hline
                kumulativni\_pocet\_prvnich\_testu\_okres & integer & ne \\
                \hline
                prirustkovy\_pocet\_prvnich\_testu\_kraj & integer & ne \\
                \hline
                kumulativni\_pocet\_prvnich\_testu\_kraj & integer & ne \\
                \hline
            \end{tabular}
        \end{center}

\subsubsection*{Přehled vykázaných očkování podle profesí (očkovací místo, bydliště očkovaného)}
Potřebné pro dotaz A3, B2 a C1.
\begin{itemize}
    \item \textbf{formát}: CSV, JSON.
    \item \textbf{dostupné \href{https://onemocneni-aktualne.mzcr.cz/api/v2/covid-19/ockovani-profese.csv}{zde}}
    \item \textbf{struktura}:
\end{itemize}
        \begin{center}
            \begin{tabular}{ |l|c|c| }
                \hline
                \multicolumn{1}{|c|}{Jméno} & Datový typ & Užitečné \\
                \hline
                \hline
                datum & date & ano \\
                \hline
                vakcina & string & ano \\
                \hline
                kraj\_nuts\_kod & string & ano \\
                \hline
                kraj\_nazev & string & ano \\
                \hline
                zarizeni\_kod & string & ne \\
                \hline
                zarizeni\_nazev & string & ne \\
                \hline
                poradi\_davky & integer & ano \\
                \hline
                indikace\_zdravotnik & boolean & ne \\
                \hline
                indikace\_socialni\_sluzby & boolean & ne \\
                \hline
                indikace\_ostatni & boolean & ne \\
                \hline
                indikace\_pedagog & boolean & ne \\
                \hline
                indikace\_skolstvi\_ostatni & boolean & ne \\
                \hline
                indikace\_bezpecnostni\_infrastruktura & boolean & ne \\
                \hline
                indikace\_chronicke\_onemocneni & boolean & ne \\
                \hline
                vekova\_skupina & string & ano \\
                \hline
                orp\_bydliste & string & ne \\
                \hline
                orp\_bydliste\_kod & integer & ano \\
                \hline
                prioritni\_skupina\_kod & integer & ne \\
                \hline
                pohlavi & string & ano \\
                \hline
                zrizovatel\_kod & integer & ne \\
                \hline
                zrizovatel\_nazev & string & ne \\
                \hline
                vakcina\_kod & string & ne \\
                \hline
                ukoncujici\_davka & boolean & ne \\
                \hline
            \end{tabular}
        \end{center}

\subsubsection*{Přehled úmrtí dle hlášení krajských hygienických stanic}
Potřebné pro dotaz B2.
\begin{itemize}
    \item \textbf{formát}: CSV, JSON.
    \item \textbf{dostupné \href{https://onemocneni-aktualne.mzcr.cz/api/v2/covid-19/umrti.csv}{zde}}
    \item \textbf{struktura}:
\end{itemize}
        \begin{center}
            \begin{tabular}{ |l|c|c| }
                \hline
                \multicolumn{1}{|c|}{Jméno} & Datový typ & Užitečné \\
                \hline
                \hline
                datum & date & ano \\
                \hline
                vek & string & ne \\
                \hline
                pohlavi & string & ne \\
                \hline
                kraj\_nuts\_kod & string & ano \\
                \hline
                okres\_lau\_kod & string & ne \\
                \hline
            \end{tabular}
        \end{center}

\subsubsection*{Epidemiologická charakteristika obcí}
Potřebné pro dotaz C1 a vlastní dotaz 1.
\begin{itemize}
    \item \textbf{formát}: CSV, JSON.
    \item \textbf{dostupné  \href{https://onemocneni-aktualne.mzcr.cz/api/v2/covid-19/obce.csv}{zde}}
    \item \textbf{struktura}:
\end{itemize}
        \begin{center}
            \begin{tabular}{ |l|c|c| }
                \hline
                \multicolumn{1}{|c|}{Jméno} & Datový typ & Užitečné \\
                \hline
                \hline
                den & string & ne \\
                \hline
                datum & date & ne \\
                \hline
                kraj\_nuts\_kod & string & ano \\
                \hline
                kraj\_nazev & string & ano \\
                \hline
                okres\_lau\_kod & string & ano \\
                \hline
                okres\_nazev & string & ano \\
                \hline
                orp\_kod & integer & ano \\
                \hline
                orp\_nazev & string & ne \\
                \hline
                obec\_kod & integer & ne \\
                \hline
                obec\_nazev & string & ne \\
                \hline
                nove\_pripady & integer & ne \\
                \hline
                aktivni\_pripady & integer & ne \\
                \hline
                nove\_pripady\_65 & integer & ne \\
                \hline
                nove\_pripady\_7\_dni & integer & ne \\
                \hline
                nove\_pripady\_14\_dni & integer & ne \\
                \hline
            \end{tabular}
        \end{center}

%Potřebná data pro dotaz A1
%https://onemocneni-aktualne.mzcr.cz/api/v2/covid-19/osoby.min.json
%https://onemocneni-aktualne.mzcr.cz/api/v2/covid-19/vyleceni.min.json
%https://onemocneni-aktualne.mzcr.cz/api/v2/covid-19/hospitalizace.min.json --pocet nove hositalizovanych
%https://onemocneni-aktualne.mzcr.cz/api/v2/covid-19/kraj-okres-testy.min.json kumulativni pocet testu

%Potřebná data pro dotaz A3
%https://onemocneni-aktualne.mzcr.cz/api/v2/covid-19/ockovani-profese.min.json

%Potřebná data pro dotaz B2
%https://onemocneni-aktualne.mzcr.cz/api/v2/covid-19/osoby.min.json
%https://onemocneni-aktualne.mzcr.cz/api/v2/covid-19/ockovani-profese.min.json
%https://onemocneni-aktualne.mzcr.cz/api/v2/covid-19/umrti.min.json


%Potřebná data pro dotaz pro C1
%https://www.czso.cz/documents/62353418/143522504/130142-21schema043021.json/07ce4a89-97ab-4da9-a8f8-c7c403d9c4b9?version=1.1  -- pocty obyvatel podle vekovych skupin v okresech a krajich
%https://onemocneni-aktualne.mzcr.cz/api/v2/covid-19/ockovani-profese.min.json -- zjisteni ockovanych lidi
%https://onemocneni-aktualne.mzcr.cz/api/v2/covid-19/osoby.min.json
%https://onemocneni-aktualne.mzcr.cz/api/v2/covid-19/obce.min.json

%Vlastni dotaz1
%https://onemocneni-aktualne.mzcr.cz/api/v2/covid-19/osoby.min.json
%https://www.czso.cz/documents/62353418/143522504/130142-21schema043021.json/07ce4a89-97ab-4da9-a8f8-c7c403d9c4b9?version=1.1
%https://onemocneni-aktualne.mzcr.cz/api/v2/covid-19/obce.min.json

%Vlastni dotaz2
%https://onemocneni-aktualne.mzcr.cz/api/v2/covid-19/osoby.min.json


\subsubsection{Český statistický úřad}
\label{data3}
Zdroj je dostupný \href{https://www.czso.cz/csu/czso/obyvatelstvo-podle-petiletych-vekovych-skupin-a-pohlavi-v-krajich-a-okresech}{zde}. Poskytuje pouze 1 dílčí sadu dat.

\subsubsection*{Obyvatelstvo podle pětiletých věkových skupin a pohlaví v krajích a okresech}
Potřebné pro dotaz C1 a vlastní dotaz 1.
\begin{itemize}
    \item \textbf{formát}: CSV.
    \item \textbf{dostupné  \href{https://www.czso.cz/documents/62353418/143522504/130142-21data043021.csv/760fab9c-d079-4d3a-afed-59cbb639e37d?version=1.1}{zde}}
    \item \textbf{struktura}:
\end{itemize}
        \begin{center}
            \begin{tabular}{ |l|c|c| }
                \hline
                \multicolumn{1}{|c|}{Jméno} & Datový typ & Užitečné \\
                \hline
                \hline
                idhod & string & ne \\
                \hline
                hodnota & integer & ano \\
                \hline
                stapro\_kod & string & ne \\
                \hline
                pohlavi\_cis & string & ne \\
                \hline
                pohlavi\_kod & string & ne \\
                \hline
                vek\_cis & string & ne \\
                \hline
                vek\_kod & string & ne \\
                \hline
                vuzemi\_cis & string & ne \\
                \hline
                vuzemi\_kod & string & ne \\
                \hline
                casref\_do & date & ne \\
                \hline
                pohlavi\_txt & string & ne \\
                \hline
                vek\_txt & string & ano \\
                \hline
                vuzemi\_txt & string & ano \\
                \hline
            \end{tabular}
        \end{center}

\subsection{Předzpracování dat}
\label{predzpracovani}
Veškeré datové sady, které jsou potřeba pro řešení dotazů v kapitole \ref{dotazy}, jsou k dispozici ve formátu CSV. Tento formát je také paměťově úspornější než formát JSON. Z těchto důvodů jsou veškerá data stahována v CSV a před vložení do databáze transformována na JSON, protože MongoDB přijímá pouze JSON respektive BSON. Při samotné transformaci jsou současně odstraňována nepotřebná data (data, která mají ve sloupci "Užitečné" napsáno "ne").

Stažení, transformaci i odstranění dat zajišťuje python modul \texttt{src/data\_handler.py}. Pro samotnou transformaci je využito knihovny \texttt{Pandas}. Výhodou této knihovny je rychlost transformace, nevýhodou vysoká spotřeba RAM u delších JSON souborů (až 10 GB). Alternativou by bylo využití knihoven \texttt{csv} a \texttt{json} a iterování přes řádky csv souboru a postupné přidávaní do výsledného json souboru. Tento způsob téměř nezatěžuje RAM, ale je mnohonásobně pomalejší. Protože je však možné využít \texttt{swapfile} (na OS Linuxového typu), tak byl zvolen přístup s využitím knihovny \texttt{Pandas}.

\subsection{Vložení dat do databáze}
K připojení do MongoDB databáze je použita třída \texttt{MongoClient} z knihovny \texttt{pymongo}. Je nutné korektně nastavit připojovací řetězec, a to nejlépe konfigurací lokálního souboru \texttt{mongo\_secrets.py}, který lze vytvořit kopií ze stejnojmenného souboru určeného k veřejné distribuci, konkrétně pojmenovaného \texttt{mongo\_secrets.py.dist}.
Uživatel má v souboru \texttt{loader.py} k dispozici nastavení pro připojení na cloudové řešení, anebo nastavení pro připojené na lokální databázový server.
Samotné vložení již předzpracovaných dat do databáze je jednoduché a tento úkon obstarává funkce \texttt{load\_data} ze souboru \texttt{loader.py}.
Datové soubory respektive datové sady jsou vkládány postupně, vždy jedna po druhé. V tuto chvíli nejsou volány žádné další funkce (například dotazy přímo nad daty v samotné databázi).
Je vytvořen implicitní index na atribut \texttt{\_id}, jiné indexy nejsou v této fázi projektu použity.

\subsection{Návod na spuštění}
Prerekvizity, specifické vlastnosti chování, potřebná nastavení a návod na spuštění jsou popsány v dokumentaci v souboru \texttt{README.md}, který je umístěn v kořenovém adresáři projektu.

%\section{Řešení 2. části projektu}

%\newpage
%\bibliography{UPA-bib}
%\bibliographystyle{bib-styles/czplain}
\end{document}
